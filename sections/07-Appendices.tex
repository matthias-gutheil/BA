\cleardoublepage

\section*{Appendix}
\label{sec:appendix}
\bigskip

\begin{center}
\textbf{Appendix A}
\end{center}

\begin{table}[htb]
\centering
\tiny
        \begin{tabular}{|c|l|l|l|}
        \hline
        \multicolumn{1}{|l|}{} & \multicolumn{1}{c|}{\textbf{Projects}}                                                                                                                                                    & \multicolumn{1}{c|}{\textbf{Programs}}                                                                                                                                                                 & \multicolumn{1}{c|}{\textbf{Portfolios}}                                                                                                                                                                                      \\ \hline
        Scope                  & \begin{tabular}[c]{@{}l@{}}Projects have defined\\ objectives. Scope is progres-\\ sively elaborated throughout the\\ project life cycle.\end{tabular}                                    & \begin{tabular}[c]{@{}l@{}}Programs have a larger scope\\ and provide more significant\\ benefits.\end{tabular}                                                                                        & \begin{tabular}[c]{@{}l@{}}Portfolios have an organizational\\ scope that changes with the\\ strategic objectives of the\\ organization.\end{tabular}                                                                         \\ \hline
        Change                 & \begin{tabular}[c]{@{}l@{}}Project managers expect change\\ and implement processes to\\ keep change managed and\\ controlled.\end{tabular}                                               & \begin{tabular}[c]{@{}l@{}}Program managers expect\\ change from both inside and\\ outside the program and are\\ prepared to manage it.\end{tabular}                                                   & \begin{tabular}[c]{@{}l@{}}Portfolio managers continuously\\ monitor changes in the\\ broader internal and external\\ environment.\end{tabular}                                                                               \\ \hline
        Planning               & \begin{tabular}[c]{@{}l@{}}Project managers progressively\\ elaborate high-level information\\ into detailed plans throughout\\ the project life cycle.\end{tabular}                      & \begin{tabular}[c]{@{}l@{}}Program managers develop the\\ overall program plan and create\\ high-level plans to guide\\ detailed planning at the\\ component level.\end{tabular}                       & \begin{tabular}[c]{@{}l@{}}Portfolio managers create and\\ maintain necessary processes\\ and communication relative to\\ the aggregate portfolio.\end{tabular}                                                               \\ \hline
        Management             & \begin{tabular}[c]{@{}l@{}}Project managers manage the\\ project team to meet the project\\ objectives.\end{tabular}                                                                      & \begin{tabular}[c]{@{}l@{}}Program managers manage the\\ program staff and the project\\ managers; they provide vision\\ and overall leadership.\end{tabular}                                          & \begin{tabular}[c]{@{}l@{}}Portfolio managers may manage\\ or coordinate portfolio\\ management staff, or program\\ and project staff that may have\\ reporting responsibilities into\\ the aggregate portfolio.\end{tabular} \\ \hline
        Success                & \begin{tabular}[c]{@{}l@{}}Success is measured by product\\ and project quality, timeliness,\\ budget compliance, and degree\\ of customer satisfaction.\end{tabular}                     & \begin{tabular}[c]{@{}l@{}}Success is measured by the\\ degree to which the program\\ satisfies the needs and benefits\\ for which it was undertaken.\end{tabular}                                     & \begin{tabular}[c]{@{}l@{}}Success is measured in terms\\ of the aggregate investment\\ performance and benefit\\ realization of the portfolio.\end{tabular}                                                                  \\ \hline
        Monitoring             & \begin{tabular}[c]{@{}l@{}}Project managers monitor and\\ control the work of producing\\ the products, services, or results\\ that the project was undertaken\\ to produce.\end{tabular} & \begin{tabular}[c]{@{}l@{}}Program managers monitor\\ the progress of program\\ components to ensure the\\ overall goals, schedules, budget,\\ and benefits of the program will\\ be met.\end{tabular} & \begin{tabular}[c]{@{}l@{}}Portfolio managers monitor\\ strategic changes and aggregate\\ resource allocation,\\ performance results, and risk\\ of the portfolio.\end{tabular}                                               \\ \hline
        \end{tabular}
    \caption[Comparative Overview of project, program and portfolio management]{Comparative Overview of project, program and portfolio management. Adapted from A Guide To The Project Management Body Of Knowledge (PMBOK Guides) (p. 8), by Project Management Institute, 2013, Pennsylvania: Project Management Institute.}
\label{tab:pmbok}
\end{table}

\clearpage
\begin{center}
\textbf{Appendix B}
\end{center}

\begin{table}[!htb]
\captionsetup{font=small}
\centering
\footnotesize
    \begin{tabular}{| l | c | c |}
    \hline
    {\bf Attributes used to classify projects} & {\bf Frequency} & {\bf Percentage} \\
    \hline
    Application area & 67 & 56 \\
    Nature of work & 54 & 45 \\
    Client/customer & 50 & 42 \\
    Complexity & 50 & 42 \\
    Cost & 48 & 40  \\
    Size & 43 & 36 \\
    Strategic importance & 39 & 33 \\
    Risk level & 35 & 29 \\
    Organisational benefit & 35 & 29 \\
    Deliverables & 35 & 29 \\
    Priority & 34 & 29 \\
    Contract type & 33 & 28 \\
    Impact & 28 & 24 \\
    Funding source & 28 & 24 \\
    Familiarity & 28 & 24 \\
    Project phase & 27 & 23 \\
    Resources & 27 & 23 \\
    Technology & 26 & 22 \\
    Clarity of goals/objectives & 25 & 21 \\
    Time & 22 & 18 \\
    Discipline & 21 & 18 \\
    Geographical location & 21 & 18\\
    Time critical & 21 & 18 \\
    Risk Type & 21 & 18 \\
    Sector/industry & 18 & 15 \\
    Organisational involvement & 17 & 14\\
    Technological uncertainty & 17 & 14 \\
    Customer involvement & 16 & 13 \\
    Client relationship & 12 & 10 \\
    Payment terms & 12 & 10 \\
    Key project success factor & 10 & 8 \\
    Stage in project life cycle & 10 & 8 \\
    Project manager & 10 & 8 \\
    Risk control & 9 & 8 \\
    Do not classify & 5 & 4\\
    Market uncertainty & 3 & 3 \\
    Regulatory/compliance & 2 & 2\\
     \hline 
    \end{tabular}
    
    \caption[Attributes used to categorise projects]{Attributes used to categorise projects. Adapted from Project Categorization Systems: Aligning Capability with Strategy for Better Results (p. 51), by Crawford, L., Hobbs, B. and Turner, J.R., 2005, Pennsylvania: Project Management Institute.}
\label{tab:cate1}
\end{table}