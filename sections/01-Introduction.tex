\cleardoublepage

\section{Introduction}
\label{sec:Introduction}
For quite a while now, project-based jobs and organisations are on the rise. This development is fuelled by the fact that many companies are struggling to cope with the accelerating changes triggered by globalisation and its byproduct digitisation, which requires them to be more agile and flexible. 
... [MISSING]

\subsection{Rational and motivation}
This thesis is part of the \textit{Project Management institute (PMI)} funded research project 'careers@projects', which is collaboratively conducted by a team of researchers from the \textit{Department of Strategy and Innovation} of the \textit{Vienna University of Economics and Business (WU)} and from the \textit{Human Resource Management and Employment Relations Faculty} of the \textit{University College Dublin (UCD)}. Its comprehensive objectives are to research the career making activities of project professionals on projects, to analyse challenges and opportunities to acquire career resources on projects and to ultimately describe implications and make recommendations for project professionals and companies interested in improving understanding of careers on projects. The methodology used by the researchers to achieve these goals is a knowledge co-creation process with project professionals, consisting of qualitative in-depth interviews and focus group workshops. As a research assistant the author of this thesis was involved in all activities of the research project, during its pilot phase. Therefore, he had the opportunity to not only gain an in-depth understanding of the subject matter, but also to evaluate a first version of his working framework in a focus group workshop with project professionals.

\subsection{Objectives and research questions}

\noindent {\bf Objectives}\\[.1cm]
Based on the before described exploratory context of this thesis, the  main objective is to develop a framework, which can be used to analyse the careers of project professionals regarding: at what stage, in what matter and why they change between project types or industries in their career. Moreover, the so far gathered data shall be used to draw conclusions and to propose adjustments and potential fields for future research to analyse. Lastly, adjustments to the preexisting questionnaire are to be made, in order to provide the required data for the working model. Above all, this bachelor thesis' goal is to contribute to the higher goal of the research project, which is to gain a deeper understanding of the careers of project professionals.\\

\noindent {\bf Research questions}\\ [.1cm]
In the context of the complex research field project management and the wide range of the overarching research project 'careers in projects' it was important to narrow down the scope of this work. Therefore, and also in respect to the research-objective, the research questions are defined as follows: 
\begin{itemize}[itemsep=0.05cm]
    \item What are patterns in the careers of project professionals after 5 | 10 | 15 | 20 years?
    \item How often do project professionals change project types and industries?
    \item What are patterns behind these changes?
\end{itemize}
\clearpage

\subsection{Research methodology}
The research methodology of this thesis and the overarching research project is based on the 4D model introduced by \citeA{maylor17}.
It consists of four steps, being \textit{D1 - Defining}, \textit{D2 - Designing}, \textit{D3 - Doing} and \textit{D4 - Describing}, which are aimed at providing guidance and structure to students conducting business and management research. In the following the research methodology of this thesis is explained using the before named, four steps of the 4D model. \\

\subsubsection{D1 - Defining}
The first step, Defining, deals with understanding academic research itself, generating and clarifying ideas and using sources of information. Based on their model, \citeA[p. 6]{maylor17} define research as 
\begin{quote}
"A systemic process that includes defining, designing, doing and describing an investigation into a particular problem."
\end{quote}
The general problem from the business and management field that is dealt with in the scope of this the research project, careers@projects, are the career developments of project professionals. The particular problem, which is derived from this general problem and that is investigated in this thesis, consists in patterns in careers of project professionals respectively project contents and industries. With reference to the stage of defining, the main difficulties faced are to differentiate the group of project professionals from other professionals, to narrow the investigative vision to the problems related to project management and to accumulate a solid theoretical basis for the final framework. Therefore, in order to deal with this complexity, an extensive literature review in relation to all relevant fields was conducted, including key words like \textit{project professionals}, \textit{project types}, \textit{project industries}, \textit{project personnel}, \textit{project classification}, \textit{project categorisation}, etc.. \\
At this point, apart from identifying these directly to the thesis related problems, the overarching research project's contexts should be looked at, in order to gain a better understanding of important correlating fields and issues. The research project is based on the concept of boundaryless careers, which captures the phenomenon of recent years that careers become more flexible, dynamic and the shift away from the organisation taking responsibility of its employees career, towards one by himself/herself having to take responsibility of one's own career \cite{inkson09}. This development challenges not only, but especially project-based organisations to develop new career paths and options in order to maintain skilled personnel. Therefore, this research is not only relevant from a academic viewpoints, but also from a practical one. So as to be able to develop such bespoke measures, organisations need to understand what underlying patterns influence and shape the careers of project professionals. Finding these patterns and explaining them is the purpose of the research project.\\

\subsubsection{D2 - Designing}
In the second part, Designing, the research approach itself is dealt with, as well as the means of conducting qualitative, quantitative and case study research \cite{maylor17}. \\
%DEFINITION OF RESEARCH APPROACH AND THE THREE APPROACHES\\
The research approach selected for this research project is qualitative as defined by \citeA{yin15}. It was selected due to the socio-scientific perspective taken in this study and implemented by the repeated execution of the following research methods with 60 project professionals: 
\begin{itemize}[itemsep=0.05cm]
    \item Indirect data collection: Curriculum vitae (CV)
    \item Observation: Systematic constellation and ground anchor
    \item Interview/Discussion: In-depth interviews
    \item Participating: Focus group workshops
\end{itemize}
The data collection methods incorporate increasing levels of personal involvement with the subject and thereby ensure a significant and sufficient database \cite{maylor17}. At this point should be noted that of the 60 interviews conducted the data of 20 was available in July 2018 in order to test the developed working framework. The other 40 interviews were conducted, but the interviews were not transcribed and therefore could not be analysed.\\
In the following each of the applied research methods will be explained in depth.\\

\noindent {\textit{Curriculum Vitae}}\\[.1cm]
As a result of the research project being in its pilot phase, the request of a CV from each candidate had, on the one hand, the purposes to gain basic information, like birth date, nationality, etc.,  from the interviewee. Furthermore, it was aimed at obtaining data regarding the roles conducted in each company and the projects performed. However, in the course of the performance of the interviews, a restriction regarding this data collection method was discovered, which constitutes in the fact that many project professionals who have worked for a long period of time in the same company have not had the need to make up a CV, as they did not apply for any position. Therefore, the necessity for a questionnaire  was identified. Developing it, considering also the data needed for the developed framework, is one of the goals of this thesis.\\ 

\noindent {\textit{Systematic constellation and ground anchor}}\\[.1cm]
The interviews were enriched with a ground anchor method, a variation of the systematic constellation method \cite{huemann16}, in which the interviewee visualises his careers and other important events, resources or projects by placing differently coloured and/or shaped cards on the ground. There were no rules, nor restrictions in regards to the degree of detail, use of colours, etc.. The goal was to help the candidate visualising and reflecting on his career during the interview, as he walked through his constellation during the interview. Moreover, the forms created were photographed to be used as data themselves in future phases of the research project. \\

\noindent {\textit{In-depth interviews}}\\[.1cm]
The interview was conducted as a one-on-one, semi-structured, personal interview \cite{maylor17}. Furthermore, it was split in three parts. In the first part, the interviewee walked through his created formation, describing for each step his/her role, whether it had been a project, the duration, the complexity and the impact on his/her career and in general if there had been a reason behind the selection of colours and shapes. In the second part, the project professional once again steps on all stations, this time explaining the competences gained or lessons drawn from each position/project. Finally, the interviewee is asked to reflect on his career as whole. The majority of the interviews were conducted in German, which required additionally to the transcription, the translation of the transcripts. Due to the number and length of each of the interviews, the translations were made using an advanced, machine-learning based online service.\\


\noindent {\textit{Focus group workshops}}\\[.1cm]
The focus groups conducted by the research team met all defined characteristics that should be inherent to this type of workshop. These criteria are "(1) a small group of people, who (2) possess certain characteristics, (3) provide qualitative data (4) in a focused discussion (5) to help understand the topic of interest."\cite{krueger14}
As of July 2018, two focus groups were conducted in the scope of the research project. The first one was in November 2017 in Munich, at an early stage of the exploratory study. It was aimed at collecting information on hindering factors (obstacles) and supporting factors for project careers in the context of infrastructure projects. It was attended by 11 members of the NETLIPSE network. The second workshop took place in May 2018 in Vienna and was targeted at evaluating and receiving feedback on first models of the final frameworks, one of them being the one developed in this thesis. Moreover, first results were presented to the 7 participants. \\

\subsubsection{D3 - Doing}
The third part, Doing, focuses on considerations to be taken in doing research, the description and analysis of the using statistical tests and the interpreting of words and actions \cite{maylor17}.\\
One crucial aspect regarding this part is, gaining access to people and organisations, which are willing to participate in the study. Since participants with specific requirements, respectively the project professional criteria, were needed for the study, one of the main means was leveraging on contacts of the researchers involved, like the connection to Austria's project management association \textit{PMA} or people known through prior research projects. In the first part of the project, first and foremost warm contacts, which are people with whom a preexisting connection exists \cite{maylor17}, were invited to participate. In later stages also a connection to cold contacts, which are people unknown to you and with whom you have nothing in common \cite{maylor17}, was established. As a result, a total of 60 interviews were conducted, of which the qualitative data of 20 are the basis to this thesis. Out of these 20, 17 are male. Moreover, the median age is 49 years, the median professional experience amounts to 22,5 years and the majority (65 per cent) has an IT background. Apart from this short introduction, the sample group will be introduced in detail in the \nth{4} chapter. The key challenges regarding the analysis of qualitative data like this are according to \citeA{maylor17}:
\begin{itemize}[itemsep=0.05cm]
    \item The data is not process nor transformed
    \item The data takes many forms
    \item The data is not standardised
    \item The data is voluminous 
\end{itemize}
Furthermore, it is crucial to ensure that the data is traceable, reliable and complete. Consequently, the main challenge of qualitative research is to derive the most important information from all data available and to insert it into a form (e.g. a matrix), which enables the research to analyse it \cite{maylor17}.\\
Accordingly,  so as to meet the challenges and to achieve a correspondence with other qualitative requirements, a analysis matrix was developed. The approach to the development of the matrix was both structured and unstructured, in the sense that the research instruction provided a formal structure for the matrix, which was extended and adapted based on the information obtained from the data. The matrix was made using Microsoft Exel. It is split into two parts, each displayed in their own spreadsheet. The first one contains an analysis of the basic data of each interviewee, including interview ID, age, gender, current industry, current company, type of employment, role and country of employment. The second one analyses the career of each project professional, starting with the first named full-time profession in the CV. From there an analysis of the indicators is made every 5 years, respectively after 5 | 10 | 15 | 20 | 25 years. The attributes analysed in each time span and for each interviewee are:
\begin{itemize}[itemsep=0.05cm]
    \item Total number of projects conducted
%    \item Average project scope (\euro)
    \item Number of different project industries worked in
    \item Name(s) of the industry(/ies) worked in (from a default list)
    \item Number of different project contents worked in
    \item Name(s) of the project content(s) worked in (from a default list)
    \item Number of company changes
\end{itemize}
The names of the project industries/contents are the only indicators that were answered in free text. For the analysis they were grouped, based on preexisting basic research, which is presented in depth in the following chapter. As the numeric data, such as the number of projects, is also in some cases deduced from non-numeric data, it relies also on the definitions and concepts presented in chapter \ref{sec:LiteratureRev}. Apart from these aspects, it has to be pointed out, that the necessity for some of the indicators was discovered during the analysis process. Furthermore, as the in-depth interviews were conducted in an semi-structured way and as a result not for all data sets is information available. Moreover, the analysis was restricted by the fact that some interviewees could not share all required information, due to their company's non-disclosure policy or they simply just did not know. Above all, a restricting factor was the general aspect that the interviews were not conducted specifically for this thesis, and therefore they are not able to cover all aspects of the for the answering of the research question required data.

\subsubsection{D4 - Describing}
The forth and last part, Describing, covers the topics making sense of the findings and presenting, reflecting and learning from the research conducted \cite{maylor17}. \\
In order to provide the reader with a guiding thread throughout the entire thesis, before the start of the actual writing process, a logically structured table of contents was developed and visualised, listing important aspects that need to be covered within each of the structure's points. Moreover, the thesis introduces the reader to the topic and the research in a logical manner, orientating on the generic project report structure proposed by \citeA{maylor17}.

\subsection{Summarised overview of chapters}
The first chapter provides an introduction to the thesis, by firstly elaborating the motivation and rational before describing its objectives and the applied research methodology. In the following chapter the foundations are laid in the means of a literature review dealing with relevant terms and concepts. In the third chapter the actual working framework will be explained, alongside with the considerations taken during the development process. In the forth chapter, after a introduction to the conducted study itself, the study's results are presented and subsequently discussed in the fifth chapter. Lastly, in the final chapter, conclusions are drawn from the findings and recommendations proposed for future research.
